% Metódy inžinierskej práce

\documentclass[10pt,twoside,slovak,a4paper]{article}

\usepackage[slovak]{babel}
%\usepackage[T1]{fontenc}
\usepackage[IL2]{fontenc} % lepšia sadzba písmena Ľ než v T1
\usepackage[utf8]{inputenc}
\usepackage{graphicx}
\usepackage{url} % príkaz \url na formátovanie URL
\usepackage{hyperref} % odkazy v texte budú aktívne (pri niektorých triedach dokumentov spôsobuje posun textu)

\usepackage{cite}
%\usepackage{times}

\pagestyle{headings}

\title{Cyberpunk 2077\thanks{Semestrálny projekt v predmete Metódy inžinierskej práce, ak. rok 2022/2023, vedenie: Ilia Sukhina}} % meno a priezvisko vyučujúceho na cvičeniach

\author{Ilia Sukhina\\[2pt]
	{\small Slovenská technická univerzita v Bratislave}\\
	{\small Fakulta informatiky a informačných technológií}\\
	{\small \texttt{xsukhina@stuba.sk}}
	}

\date{\small 27. september 2022} % upravte



\begin{document}

\maketitle

\begin{abstract}
	Cyberpunk 2077 je adventúra pre jedného hráča od štúdia CD Projekt Red, ktorá sa odohráva v roku 2077 vo fiktívnom americkom meste Night City, kde vyrastal náš hlavný hrdina Vi. V tomto kyberpunkovom meste sa naša hlavná postava, ktorá sa snaží uniknúť a stať sa legendou Night City, dostane na misiu, ktorá mu mala priniesť veľkú slávu, ale pre neúspech musí neskôr bojovať o život s osobnosťou bývalého terorista-rockera v hlave. Tento článok vám povie o tom, prečo vydanie hry zlyhalo, histórii cyberpunkového vesmíru, zápletke cyberpunku, zaujímavej mechanike a ich implementácii, o tom, ako hra ovplyvnila samotný žáner cyberpunku a ako CD Projekt Red vrátil popularitu svojej hre svojím brilantným marketingovým ťahom.
\end{abstract}



\section{Úvod}

Tento článok vysvetlí, čo je Cyberpunk 2077 a prečo je to neobvyklá hra na zbieranie peňazí. Povie vám, prečo hra zlyhala vo vydaní, posolstvo a históriu hry, jej mechaniku, vplyv na samotný žáner cyberpunk a ako CD Project RED nedávno získal popularitu pre svoju hru. Táto téma bola zvolená kvôli tomu, že táto hra sa stala najkontroverznejšou a najškandalóznejšou hrou roku 2022, okolo ktorej vzrušenie vzrástlo viac ako v ktorejkoľvek inej hre za posledné desaťročia, a to vďaka tomu, že sa hra vyplatila iba na predpredaj a napriek svojej nedokonalosti pri vydaní ju stále milovali mnohí hráči pre jej atmosféru, neobvyklú zápletku a novú hernú mechaniku.


\section{Vydanie} \label{vydanie}

Cyberpunk 2077 bol najočakávanejším vydaním roku 2020, ale bol to najväčší neúspech. Ale prečo? Prvýkrát bol cyberpunk spomenutý na výstave E3 v roku 2012 spolu s oznámením novej hry "The Witcher 3", potom sa o hre nehovorilo mnoho rokov. V roku 2018 CD Project RED prvýkrát ukázal nový teaser hry, v ktorej ukázal svet hry a postáv a uviedol dátum vydania 16.Apríla 2020, hra sa divákom tak páčila, že bola považovaná za nového záchrancu priemyslu. Na E3 2019 na prekvapenie všetkých sám Keanu Reeves predstavil hru, čo ešte viac podnietilo záujem o hru. Ale bližšie k dátumu vydania sa spoločnosť rozhodla odložiť dátum vydania najskôr o mesiac kvôli "vylepšeniam" a potom o ďalšie dva mesiace a potom o ďalší mesiac si všetci začali uvedomovať, že niečo nie je v poriadku. Napriek tomu, keď sa rozhodli vydať hru 10.decembra, všetko sa zdalo v poriadku, nový online rekord na Steame, milióny predaných kópií, ale prvý deň sa hráči stretli s absolútne nepripravenou hrou. Hra havarovala každú pol hodinu, grafika bola o úroveň nižšia, ako je uvedené na prezentáciách, textúry odleteli z objektov, kopa chýb, ktoré jednoducho nedovolili prejsť dejom a veľmi zlá optimalizácia. To všetko znížilo podiely CD Project RED a hra sa stala naj katastrofálnejším projektom podľa očakávaní, aj keď napriek tomu veľa hráčov hru obhájilo s odvolaním sa na príklad hry "Stalker", ktorá bola pri vydaní rovnaká nepripravená, ale potom bola hra dokončená a táto hra sa stala kultom.
A kto je zodpovedný za toto zlyhanie? Keďže nie je ťažké uhádnuť, ten, kto dal prvý dátum vydania, je na vine, a to Vedenie spoločnosti. Bližšie k dátumu vydania si všetci začali uvedomovať, že sú veľmi pozadu. Preto vedenie zaviedlo prácu nadčas. Najprv dobrovoľné a potom dobrovoľné-povinné. Ľudia pracovali 16 hodín denne takmer sedem dní v týždni. A aj napriek tomu hra vyšla nepripravená.  Tomu všetkému by sa dalo predísť, keby vedenie lepšie pochopilo, v akom štádiu vývoja sa hra nachádzala, a konzultovalo s vývojármi, a nie samostatne rozhodovať o tom, čo je pripravené a čo nie.



\section{História Cyberpunk 2077} \label{historia}




\subsection{Nejaké vysvetlenie} \label{ina:nejake}

Niekedy treba uviesť zoznam:

\begin{itemize}
\item jedna vec
\item druhá vec
	\begin{itemize}
	\item x
	\item y
	\end{itemize}
\end{itemize}

Ten istý zoznam, len číslovaný:

\begin{enumerate}
\item jedna vec
\item druhá vec
	\begin{enumerate}
	\item x
	\item y
	\end{enumerate}
\end{enumerate}


\subsection{Ešte nejaké vysvetlenie} \label{ina:este}

\paragraph{Veľmi dôležitá poznámka.}
Niekedy je potrebné nadpisom označiť odsek. Text pokračuje hneď za nadpisom.



\section{Dôležitá časť} \label{dolezita}




\section{Ešte dôležitejšia časť} \label{dolezitejsia}




\section{Záver} \label{zaver} % prípadne iný variant názvu

\section{Iny zaver}



%\acknowledgement{Ak niekomu chcete poďakovať\ldots}


% týmto sa generuje zoznam literatúry z obsahu súboru literatura.bib podľa toho, na čo sa v článku odkazujete
\bibliography{literatura}
\bibliographystyle{alpha} % prípadne alpha, abbrv alebo hociktorý iný
\end{document}
