% Metódy inžinierskej práce

\documentclass[10pt,twoside,english,a4paper]{article}

\usepackage[english]{babel}
%\usepackage[T1]{fontenc}
\usepackage[IL2]{fontenc} % lepšia sadzba písmena Ľ než v T1
\usepackage[utf8]{inputenc}
\usepackage{graphicx}
\usepackage{url} % príkaz \url na formátovanie URL
\usepackage{hyperref} % odkazy v texte budú aktívne (pri niektorých triedach dokumentov spôsobuje posun textu)

\usepackage{cite}
%\usepackage{times}

\pagestyle{headings}

\title{Cyberpunk 2077\thanks{Semestrálny projekt v predmete Metódy inžinierskej práce, ak. rok 2022/2023, vedenie: }} % meno a priezvisko vyučujúceho na cvičeniach

\author{Ilia Sukhina\\[2pt]
	{\small Slovenská technická univerzita v Bratislave}\\
	{\small Fakulta informatiky a informačných technológií}\\
	{\small \texttt{xsukhina@stuba.sk}}
	}

\date{\small 27. september 2022} % upravte



\begin{document}

\maketitle

\begin{abstract}
	Cyberpunk 2077 is a single-player adventure game from studio CD Projekt Red, set in the year 2077 in the fictional American city of Night City, where our protagonist V grew up. In this cyberpunk city, our protagonist, trying to escape and become a Night City legend, gets on a mission that was supposed to bring him great fame, but due to failure he later has to fight for his life with the personality of a former terrorist rocker in his head. This article will cover why the cyberpunk release failed, the history of the cyberpunk universe, the cyberpunk story, interesting mechanics and their implementation, how the game influenced the cyberpunk genre itself and how CD Projekt Red with its genius marketing move brought its game back to popularity.
\end{abstract}



\section{Introduction}

This topic was chosen because this game has become the most controversial and scandalous game of 2022, which has raised more hype than any other game in the last decades, because of the fact that the game has a very interesting universe which is interesting to follow and despite its unfinished nature at release it was still loved by many players for its atmosphere, unusual plot and new mechanics.  This article will deal with when and how the game was introduced, why it failed in release, the universe story and what inspired its creator, details about the game's story and the characters in it, gameplay, the game's problems, "braindances", cyberspaces and how they are arranged, the impact of Cyberpunk 2077 on the genre and how CD Projekt Red with its genius marketing move brought its game back to popularity.


\section{Game release} \label{game release}

	Cyberpunk 2077 was the most anticipated release of 2020, but was the biggest fail. Why? Cyberpunk was first mentioned at E3 expo in 2012 along with the announcement of the new game “The Witcher 3”, it grabbed the attention of thousands of gamers, everyone started imagining what the game will be, but after that expo studio did not talk about this game for years. In 2018 CD Project RED first showed a new teaser of the game in which it showed the game world, characters and indicated a release date of April 16, 2020, the game was so liked by the audience that it became considered the new saviour of the industry, because many thought that a studio that released such a good game as “The Witcher 3” could not disappoint. At E3 2019, to everyone's surprise, the famous actor Keanu Reeves made a presentation on the game, whose character would be punk-rocker, which further fuelled interest in the game. But closer to the release date the company decided to postpone the game's release first for a month for a little "improvments", and then for another two months, and then for another month, everyone began to realize that something was wrong. Still, when the game was decided to be released on December 10, everything seemed fine, a new online record on Steam, millions of copies sold, but on the very first day the players were met with a completely unready game. The game crashed every half an hour, the graphics was on the lower level than shown in the presentations, textures were flipping from objects, a lot of bugs that just didn't allow to pass the storyline and very bad optimization. All this caused the stocks of CD Project RED to plummet and the game became the worst project of the year, based on expectations, although many players defended it, giving the example of "Stalker", which was just as unfinished on release, but then the game was improved and it became a cult one.
And who is responsible for this failure? As you can easily guess, the one to blame is the one who gave the first release date, namely the company's management, because even the programmers at the time did not know when the game was going to be released, which was incredibly shocking for them. Closer to the release date, everyone began to realize that they were very far behind schedule. Therefore, the management introduced overtime, first voluntarily and then mandatory. People worked 16 hours a day with practically no days off. And even so, the game was not finished in time.  All of this could have been avoided if the management had understood better what stage of development the game was in and had consulted with the developers rather than deciding for themselves what was ready and what was not. This whole debacle had a huge impact on the game as a whole.
\cite {RePEc:hal:journl:hal-03633678}


\section{History of the Cyberpunk} \label{historia}

	Cyberpunk was originally a board game that had 3 versions, but the most popular was Cyberpunk 2020. The story of Cyberpunk 2077 is taken directly from this board game. The name of the game itself refers both to the genre and literally to the fact that the protagonist is a cybernated punk (Cyber Punk). In creating the universe, the author, namely Michael Pondsmith was inspired by the movie “Blade Runner” and many books in the same genre. In the universe, as well as in the game, the genre of music such as 'Rock' is elevated to an absolute. In the board game, next to the description of the game's mechanics, the main rules of behaviour are spelled out - "Style is more important than meaning, cool is everything. Go to the limit, break the rules", which is literally the motto of any punk.
	At the heart of the cyberpunk story is the conflict between corporations and the lower classes of society. Generally such a phenomenon as the incredible class distinction in society is the main idea of the whole universe, taken from Japanese culture. The game is very much connected to Japanese culture, hence the musical group Samurai, the Arasaka Corporation, amount of Japanese migrants and a huge number of Japanese-style streets with neon signs.
	The main story begins in the early 1990s. The Soviet Union falls apart, but liberation from the forced teaching of the ideology of "Communism" and the loss of the Baltics still saves the country from total collapse. However, the authority of the USSR remained strong on the world stage and Gorbachev's successor was able to get closer to Europe, which further strengthened the country's position. At the same time, Europe decided to switch to a single currency, the “eurodollars”, against the huge number of migrants, which was later followed by the whole world. At the same time, Europe and the Soviet Union set up a joint space exploration and exploration programme and created a colony on the moon.
Things are bad in America at the time. The CIA, NSA, FBI and DEA teamed up and organised the "Gang of Four" and brought all power to bear by assassinating the incumbent president and vice-premiers. In the end, such power dragged the country into two wars with South America, which further destroyed the country. Later, the military decided to take over, but Texas, Alaska, California and Nevada seceded from the country and declared their independence. There was chaos all over the country but in spite of the situation a businessman Richard Night decided to build a modern futuristic city named "Caranada City" with money from investors, which later after assassination of Richard was renamed after him and became "Night City", where the game Cyberpunk 2077 takes place. In exchange for the investment, the corporations were given their parts of the city and monopolies to run it.
 In Africa, an African alliance was organised. The Middle East had become a nuclear wasteland during the wars. And Japan was at the height of development at the time and with the help of Arasaka, the country was equipped with the most advanced weapons.
	And at this moment the world moves into the cyberpunk era. In Europe, the first neural-interface implants are being created, which have direct contact with the brain, in America they are creating combat flying machines against riots, and Japan is succeeding in inventing combat robots. And then everything started developing with incredible speed, more and more implants, prosthetics that increase human capabilities, “braindances”, interfaces to enter the World Wide Web, artificial intelligence, and people began to be cloned on a massive scale. And due to the lack of market control, all these developments began to be sold freely to all comers.
	After getting rid of the “Gang of Four”, power supposedly returned to the people, but this time all power went to corporations, which were richer than the countries themselves. The corporations became a substitute for the state, they were in charge of all infrastructure, be it medicine or security, they bought out cities and created their own armies.
	In such a world, there are several major corporations:

1. Arasaka\\
A Japanese company specialising in security and weapons. The owner of this company is Saburo Arasaka, who genuinely believes that Japan is the greatest country and does everything to make it so. It is this company that has most control of Night City.

2. Militech\\
	An American company specialising in weapons and supplying these weapons to all possible armies. It has direct links with the American government. The owner of this company genuinely believes that America is the greatest country, which leads to a conflict between Arasaka and Militech.

3. Trauma Team\\
	A military medical service that will save its client at any cost. It is the only company that does this kind of work in Night City and if one does not have insurance, he is condemned to deal with the situation on he's own.

4. Delamain\\
	A taxi service controlled by artificial intelligence that will drive you and take you away from anywhere and no matter how dangerous it is. 

There are more than a dozen other corporations, but all the action is mainly related, just to these four.
 
From there, the story of the cyberpunk universe transitions into the plot of the game itself.

\subsection{Cyberpunk 2077 storyline} \label{storyline}
\subsection{Who is Johnny Silverhand?}\label{johnny}
	The 2000's came and some guy volunteered for the corporate wars, realised the absurdity of these wars and refused to die for a showdown between the corporations, deserted, returned to his hometown and decided that he will change the world, this guy called himself Johnny Silverhand. He was a real punk, a rocker, speaking out against the corporations and urging everyone to go against the authorities, through his musical band Samurai. Of course, the corporations tried their best to get rid of Johnny.
	So, one day after a concert, Johnny and his girlfriend Alt Cunningham were walking the streets, but suddenly they were attacked by mercenaries, Alt was kidnapped and Johnny was seriously injured, but lucky for Johnny a journalist was watching at this situation and quickly took him to the 'ripper' after they leaved. Johnny quickly realised that it was "Arasaka", as the mercenaries who had attacked them had military implants that normal mercenaries couldn't get their hands on. The reporter told Johnny that the "Arasaka" weren't actually trying to kill him intentionally, they were after his girlfriend. As it turned out, Alt was a "netrunner" - a computer technology professional who could navigate cyberspace data streams, she was also the creator of the "Soulkiller" - a program for digitising identities and "Arasaka" wanted to get their version of this program. Johnny decided he would save his girlfriend and hired experienced 'fixers' to help him sneak into Arasaka Taur, but after a long firefight there he found only Alt's dead body, she had transferred her consciousness to the network.
For decades Johnny has been planning his revenge. And when the chance came, with the support of another corporation, “Militech”, Johnny and his team of mercenaries used a helicopter to sneak into “Arasaka Taur” headquarters through the roof and plant a nuclear shell in a lift, which descended to the base of the building. But Johnny is contracted by “Militec” to still upload all the data from their network and put it on the internet, including the soul of Alt Cunningham and run a virus into their corporate network. Because of this, Johnny fails to fly away with everyone else and is captured by Adam Smasher, Arasaka's chief mercenary, and taken to Sabura Arasaka. After a brief conversation with Saburo about why Johnny blew up Arasaka-Taur, his own girlfriend's software is used on Johnny and his consciousness is transferred to a chip.
\subsection{The main plot of the game}\label{plot}
	57 years after this the story of V begins, there are three roles in the game that player can start playing as - Nomad, Corporate, and Street Child. These roles have almost no effect on the gameplay, as the only difference is the different prologue and the unique lines in the dialogues. The entire plot takes place in Night City in the year 2077, where our protagonist V is trying to become a Night City legend like Johnny Silverhand or David Martinez. For years, V has been doing various small jobs with his best friend Jackie Wells, but after one mission to save a human, they become popular, and one day they get a really serious order from "fixer" Dexter DeShon to steal a chip from an "Arasaka" for Evelyn Parker. But first V have to meet the customer. Meeting the customer, V gets the details of the plan with Evelyn's "braindance", edited by Judy, Evelyn's best friend and the best “braindance” editor in Night City. We also learn that Evelyn was the mistress of Yorinoba Arasaki, son of the great Sabura Arasaki, and that he stole his father's biochip from his lab to sell it to Europe because he didn't share his father's views and wanted some change in the family empire. 
	On the day of the mission, Jackie and V are taken by taxi Delamain to the hotel where Yorinobu is staying before going to Europe. After checking in and hacking into the hotel's network with a 'netrunner', they head to Yorinobu's room, where they find the biochip, but suddenly things start to go wrong, as Yorinobu's father arrives to find out that his son has stolen the chip and while talking about it, Yorinobu in a fit of rage kills his father and tells everyone that his father has been poisoned. They have to escape through the roof when Arasaka's army unit notices them and opens fire at them. Eventually, they have to jump down, Jackie gets shot and the case with the chip gets damaged, so Jackie inserts the chip into his head to save it. With Delamain's help the lads escape, but on the way to the meeting place Jackie gets worse and before he dies he hands the chip to V saying he hopes V becomes a legend. Arriving at the meeting place with Dexter, V is shot in the forehead for failing the mission, but the chip saves V's life by replacing the damaged brain cells with new ones. At the junkyard where Dexter took him, he is found by Sabura Arasaka's former bodyguard, Takemura, and taken to the “riper”. On reaching the doctor later, V finds out that the chip in his head is the soul of Johnny Silverhand and that if he will not do anything, the chip will replace all of his brain cells with Johnny's. After a while V starts to see Johnny and talk to him, now they have one brain for two. Later we learn that the bodyguard knows that V didn't kill Saburo and that he wants to avenge his master and first they need to find Anders Hellman, who is the head of the 'save your soul' project aka the 'soulkiller' new version, with the help of Johnny's old friend Rogue Amendiares.
	To do this, they head to "Afterlife", the club where the toughest mercenaries and the best “fixers” can talk about their plans. There they meet up with Rogue, who, for a small sum of money, gives V information about Hellman's location and tells him that Panam Palmer will help him. Panam is a member of family named “Aldecaldos” and they are nomads that live outside Night City. After helping Panam deal with the mistakes of the past they become good friends and even more than friend. So she help sV to get Hellman, from him V finds out that he has almost no chance of survival. After realising it he calls Takemura, and Takemura gets all information about Saburo’s daughter.
	Then Judy calls V and tells him that Evelyn is missing. After a failed attempt to steal the chip, Evelyn spent a few days in “Lizzie's Bar” under protection, but then decided she would be safer at her job, she worked as a "doll", but there through a CCTV camera she gets hacked by "Voodoo Boys" and burns the biochip in her head. So in a state of coma, she goes from hand to hand, where she is eventually sold to "scavengers" to film illegal "braindances". After Evelyn's rescue , V discovers that she used to work for the 'Voodoo boys', but after realising that biochip was quite expensive she wanted to start her life all over again by selling it to  'NetWatch'. That's why she ended up in slavery, after such information V goes to the "Voodooists" to find out what they know about the biochip and why they needed it.
	After completing a mission from the 'voodooists' to get rid of the “NetWatch”, V gets to their head, Brigitte, who is willing to help V, but as it turns out they only wanted to get behind “The Blackwall” and talk to Alt. Luckily, after crossing over, Alt recognises Johnny and after a brief conversation with Alt, she agrees to help them separate, but to do so they need to get to 'Mikoshi', a place where the souls of all those who have been swallowed by the 'soulkiller' are kept.
	Later, with Takemura's help, V is able to reach Yorinoba Arasaka's sister, Hanako Arasaka, and talk to her, after telling her the truth about who killed her father, they were attacked by Arasaka's military and V had to run, but after a while Hanako contacts him and offers to discuss further actions.\\
	At this point we are already approaching the culmination of the story, there are only 3 main endings in the game, which are divided into two and there is one secret ending. \\
The first ending is if V decides to trust Arasaka, then we help Hanako get rid of her brother by using his body as a vessel for her father's soul. So Arasaka will continue to exist for many more years, and V will eventually end up at the Arasaka Clinic in space, where V will be offered either to digitize his identity or head back to Night City and die after six months, as the operation was not very successful.\\
The second ending is if V decides to ask “Aldecaldos” for help, then they storm the “Mikoshi” while losing some friends, where Alt will offer V to  return to his body and die soon, or give his body to Johnny and become one consciousness with her.\\
The third ending is available if V had a good relationship with Johnny. Then Johnny will ask for help from Rogue, where after a firefight they will break into “Mikoshi” and Alt will offer V the same condition as in the second ending.\\
And the secret ending is if V goes to “Mikoshi” on his own. But the ending is always the same.
\section{Gameplay}\label{gameplay}
	The gameplay is not particularly linear, but very atmospheric, because an incredible amount of time and effort was invested in the artistic style of the game, in the layout of the city, in the
architecture, in the design of characters, cars, clothes, billboards, not to mention the fact that almost every object in the city has its own unique model. The authors just incredibly accurately managed to convey the whole spirit of cyberpunk culture, it’s madness and entourage of futuristic. Each character is a separate personality with its own backstory that can be even used to make a movie about. After all, for an interesting cyberpunk experience you need to understand what you are playing in, here you can't pass the game as some part of the "GTA", because there are a lot of nuances that you can't understand without a backstory.

	The game has very well implemented shooting and hacking mechanics. The shooting sensation and the sound of the weapon immerses you in the gameplay and even the recoil of the weapon made very good, the game has a smart weapon that is self-directed and in which you can select the lethality mode. Hacking mechanics in the teaser and on the release are very different, in the game hacking terminals are made using a table with the values you need to pick up, and almost all of the hacking people are done using scripts that are sent in a single click, scripts also come in, both lethal and non-lethal. In gameplay they combine very well. All connections are made through ports in the hand and head, and a neural interface, the so-called 'personal port' and 'neuroport', all of which are directly linked to the brain. These ports can be used to both upload and download files, information and viruses.

	It's also worth paying huge attention to the "side" quests that fill the whole game, these quests are divided into large and small ones. Small ones are the usual quests like go and fetch, kill someone, steal. But there are also the big ones, and they create more entourage to the game, because, for example, in one of the side quests we find out that there is a top secret organization that can change people's personalities, influence their decisions and thoughts, erase memories and add to them. All these changes to a person happen through microwaves from whatever is possible. And in another "side" quest we get an order to kill a criminal who murdered the customer's wife, but the usual mission about revenge turns into speculation about God and his designs, because the criminal believed in God and made a contract with a corporation and he was ready to sacrifice himself as Issus, recording it all on a "braindance" to show people what unconditional love is. We can either prevent this madness or do nothing. Missions like this help to immerse ourselves in the cyberpunk world even more.

	But the great thing about this game is that the whole game is a punk motto "Style is more important than meaning, awesomeness is everything. Go to the limit, break the rules", because if player follow the storyline, as the game says to, he will get dull, boring passage and will not get any bright emotions from the game, but if he breaks the rules, go against the system and do what he wants, the game opens up new interesting passing opportunities, which makes the game a simply brilliant work of art. The developers have put it in the game on purpose, because the game should be enjoyed, and if you just thoughtlessly run through only the storyline, you will never understand the depth and soul that the developers have put into it. Having completed the game several times, the game reveals itself differently each time, because getting the information from one playthrough, you understand the story of the other.
\subsection{Problems with gameplay}\label{problems}
	I would like to say right away that this section is only relevant at the time of writing, because the developers are trying very hard to fix their game. And they have already fixed most of the bugs that were on release.

1. Optimization.

At the moment an average PC in the 1-1.5 thousand euro range can handle Cyberpunk 2077 at 60-70 fps, with significant drawdowns up to 40 when entering a location with lots of reflections or neon lighting. You have to have a more expensive PC to pass comfortably, which makes it hard to get a new audience, as no one will spend several thousand euros to play any 1 particular game.

2. The scripted nature of the world.

The cyberpunk world is huge, comparable to the world in the latest 'GTA', but everything that happens in the streets of cyberpunk is just pre-written scripts without any random events, like in the same 'Red Dead Redemption'.

3. Bugs with textures

In some dialogues, certain objects may disappear, appear or freeze, which ruins the atmosphere of the game and it no longer feels as soulful.

4. The absence of any kind of branching plot.

Regardless of the options chosen in the game, we still have to go to the end of the game one way, albeit with different endings. There is no option to redirect the storyline in a different direction.

5. Absence of repercussions from the character.

Regardless of how we respond in the dialogue almost no character will stop treating us well, they may only reprimand the player for being uncivil, but later they will immediately forget about it and the dialogue will continue as if it never happened.
This game is not perfect at all, this bags really affect on gameplay, but even with them somehow you want to
\section{Impact on Cyberpunk culture}\label{culture}
\cite {Sun2022}
\begin{itemize}
\item jedna vec
\item druhá vec
	\begin{itemize}
	\item x
	\item y
	\end{itemize}
\end{itemize}

Ten istý zoznam, len číslovaný:

\begin{enumerate}
\item jedna vec
\item druhá vec
	\begin{enumerate}
	\item x
	\item y
	\end{enumerate}
\end{enumerate}


\subsection{Ešte nejaké vysvetlenie} \label{ina:este}

\paragraph{Veľmi dôležitá poznámka.}
Niekedy je potrebné nadpisom označiť odsek. Text pokračuje hneď za nadpisom.



\section{Dôležitá časť} \label{dolezita}




\section{Ešte dôležitejšia časť} \label{dolezitejsia}




\section{Záver} \label{zaver} % prípadne iný variant názvu

\section{Iny zaver}



%\acknowledgement{Ak niekomu chcete poďakovať\ldots}


% týmto sa generuje zoznam literatúry z obsahu súboru literatura.bib podľa toho, na čo sa v článku odkazujete
\bibliography{literatura}
\bibliographystyle{plain} % prípadne alpha, abbrv alebo hociktorý iný
\end{document}
